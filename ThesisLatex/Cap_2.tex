\chapter{Nombre del Cap�tulo II}

Cada cap�tulo deber� contener una breve introducci�n que describe en forma r�pida el contenido del
mismo. En este cap�tulo va el marco te�rico. (pueden ser dos cap�tulos de marco te�rico)

\section{Secci�n 1 del Cap�tulo II}

Un cap�tulo puede contener n secciones. La referencia bibliogr�fica se hace de la siguiente manera:
\cite{Mateos00}

\subsection{Sub Secci�n}

Una secci�n puede contener n sub secciones.\cite{Galante01}

\subsubsection{Sub sub secci�n}

Una sub secci�n puede contener n sub sub secciones.

\section{Recomendaciones generales de escritura}
Un trabajo de esta naturaleza debe tener en consideraci�n varios aspectos generales:

\begin{itemize}
\item Ir de lo gen�rico a lo espec�fico. Siempre hay qeu considerar que el lector podr�a ser alguien no muy familiar con el tema 
y la lectura debe serle atractiva.
\item No poner frases muy largas. Si las frases son de mas de 2 l�neas continuas es probable que la lectura sea dificultosa.
\item Las figuras, ecuaciones, tablas deben ser citados y explicados {\bf antes} de que aparezcan en el documento.
\item Encadenar las ideas. Ninguna frase debe estar suelta. Siempre que terminas un p�rrafo y hay otro a continuaci�n, 
el primero debe dejar abierta la idea que se explicar� a continuaci�n. Todo debe tener secuencia.
\end{itemize}


\section{Consideraciones Finales}

Cada cap�tulo excepto el primero debe contener al finalizarlo una secci�n de consideraciones que enlacen
el presente cap�tulo con el siguiente.
