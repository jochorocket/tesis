\begin{resumen}
La presente tesis propone un conjunto de modificaciones al algoritmo evolutivo de inspiraci�n cu�ntica para reales \ac{QIEA}-$\mathbb{R}$, las cuales se basan tanto en una segregaci�n de los campos de acci�n de los individuos cu�nticos durante su operaci�n, as� como en la implementaci�n de un operador de cruzamiento que modifique a los individuos cu�nticos a diferencia de el grueso de algoritmos que concentran dichos operadores en los individuos cl�sicos. Los algoritmos previamente mencionados han sido probados sobre un conjunto de funciones \textit{benchmark} en la b�squeda de mejoras tanto en la calidad de la data generada, como en la convergencia durante la ejecuci�n de los ya mencionados. Los resultados muestran una mejora en la velocidad de convergencia para individuos que implementan el operador de cruzamiento en escenarios con n�mero bajo de individuos cl�sicos generados por individuo cu�ntico por iteraci�n.
\end{resumen}