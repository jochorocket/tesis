\chapter{Conclusiones y Trabajos Futuros}\label{chap:conclusiones}

El presente trabajo ha propuesto distintas modificaciones al algoritmo original \ac{QIEA}$\mathbb{R}$ con la expectativa por obtener resultados mejores en t�rminos de tiempo de convergencia y calidad de datos generados. En t�rminos generales, los �nicos algoritmos que han registrado resultados medianamente consistentes y competitivos a lo largo de todas las pruebas son las variantes que implementan el operador de cruzamiento en espacios particionados. Esto es: \ac{QIEA}$\mathbb{R}$-pCO y U\ac{QIEA}$\mathbb{R}$-pCO.

Desafortunadamente, en los casos en los que registran mejoras sobre los algoritmos originales (sean \ac{QIEA}$\mathbb{R}$ y \ac{QIEA}$\mathbb{R}$-p), �sta no resulta ser muy notoria. Adem�s, hay otra cantidad similar de casos en los que ambos se ven superados, ya sea por uno u otro.

Los dem�s algoritmos propuestos obtuvieron resultados m�s irregulares y menos �ptimos, por lo que se desaconseja su consideraci�n a futuro. Cabe destacar que los algoritmos con peores resultados en general fueron los que implementaban el particionamiento del espacio de b�squeda sin el operador de cruzamiento.

\section{Problemas encontrados}
La segunda  parte de ests cap�tulo corresponde a los problemas encontrados. esta seccion es muy importante para que los siguientes estudiantes que hagan algo en esta l�nea no cometan los mismos errores y tu tesis sea un buen pelda�o para avanzar m�s r�pido.

\section{Recomendaciones}
En esta secci�n el tesista debe reflejar que la tesis ha permitido adquirir nuevos conocimientos que podr�an servir para guiar otros trabajos en el futuro.

\section{Trabajos futuros}
En base a los puntos anteriores es recomendable que tu tesis tambi�n sugiera trabajos futuros. Esta secci�n es esecialmente �til para otras ideas de tesis.

Todo este cap�tulo no debe ser m�s de unas 4 p�ginas.